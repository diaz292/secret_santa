\documentclass[]{article}
\usepackage{lmodern}
\usepackage{amssymb,amsmath}
\usepackage{ifxetex,ifluatex}
\usepackage{fixltx2e} % provides \textsubscript
\ifnum 0\ifxetex 1\fi\ifluatex 1\fi=0 % if pdftex
  \usepackage[T1]{fontenc}
  \usepackage[utf8]{inputenc}
\else % if luatex or xelatex
  \ifxetex
    \usepackage{mathspec}
  \else
    \usepackage{fontspec}
  \fi
  \defaultfontfeatures{Ligatures=TeX,Scale=MatchLowercase}
\fi
% use upquote if available, for straight quotes in verbatim environments
\IfFileExists{upquote.sty}{\usepackage{upquote}}{}
% use microtype if available
\IfFileExists{microtype.sty}{%
\usepackage{microtype}
\UseMicrotypeSet[protrusion]{basicmath} % disable protrusion for tt fonts
}{}
\usepackage[margin=1in]{geometry}
\usepackage{hyperref}
\hypersetup{unicode=true,
            pdftitle={A Diaz, Guzman, \& Co. Xmas: Secret Santa 2018},
            pdfauthor={David A. Diaz},
            pdfborder={0 0 0},
            breaklinks=true}
\urlstyle{same}  % don't use monospace font for urls
\usepackage{color}
\usepackage{fancyvrb}
\newcommand{\VerbBar}{|}
\newcommand{\VERB}{\Verb[commandchars=\\\{\}]}
\DefineVerbatimEnvironment{Highlighting}{Verbatim}{commandchars=\\\{\}}
% Add ',fontsize=\small' for more characters per line
\usepackage{framed}
\definecolor{shadecolor}{RGB}{248,248,248}
\newenvironment{Shaded}{\begin{snugshade}}{\end{snugshade}}
\newcommand{\KeywordTok}[1]{\textcolor[rgb]{0.13,0.29,0.53}{\textbf{#1}}}
\newcommand{\DataTypeTok}[1]{\textcolor[rgb]{0.13,0.29,0.53}{#1}}
\newcommand{\DecValTok}[1]{\textcolor[rgb]{0.00,0.00,0.81}{#1}}
\newcommand{\BaseNTok}[1]{\textcolor[rgb]{0.00,0.00,0.81}{#1}}
\newcommand{\FloatTok}[1]{\textcolor[rgb]{0.00,0.00,0.81}{#1}}
\newcommand{\ConstantTok}[1]{\textcolor[rgb]{0.00,0.00,0.00}{#1}}
\newcommand{\CharTok}[1]{\textcolor[rgb]{0.31,0.60,0.02}{#1}}
\newcommand{\SpecialCharTok}[1]{\textcolor[rgb]{0.00,0.00,0.00}{#1}}
\newcommand{\StringTok}[1]{\textcolor[rgb]{0.31,0.60,0.02}{#1}}
\newcommand{\VerbatimStringTok}[1]{\textcolor[rgb]{0.31,0.60,0.02}{#1}}
\newcommand{\SpecialStringTok}[1]{\textcolor[rgb]{0.31,0.60,0.02}{#1}}
\newcommand{\ImportTok}[1]{#1}
\newcommand{\CommentTok}[1]{\textcolor[rgb]{0.56,0.35,0.01}{\textit{#1}}}
\newcommand{\DocumentationTok}[1]{\textcolor[rgb]{0.56,0.35,0.01}{\textbf{\textit{#1}}}}
\newcommand{\AnnotationTok}[1]{\textcolor[rgb]{0.56,0.35,0.01}{\textbf{\textit{#1}}}}
\newcommand{\CommentVarTok}[1]{\textcolor[rgb]{0.56,0.35,0.01}{\textbf{\textit{#1}}}}
\newcommand{\OtherTok}[1]{\textcolor[rgb]{0.56,0.35,0.01}{#1}}
\newcommand{\FunctionTok}[1]{\textcolor[rgb]{0.00,0.00,0.00}{#1}}
\newcommand{\VariableTok}[1]{\textcolor[rgb]{0.00,0.00,0.00}{#1}}
\newcommand{\ControlFlowTok}[1]{\textcolor[rgb]{0.13,0.29,0.53}{\textbf{#1}}}
\newcommand{\OperatorTok}[1]{\textcolor[rgb]{0.81,0.36,0.00}{\textbf{#1}}}
\newcommand{\BuiltInTok}[1]{#1}
\newcommand{\ExtensionTok}[1]{#1}
\newcommand{\PreprocessorTok}[1]{\textcolor[rgb]{0.56,0.35,0.01}{\textit{#1}}}
\newcommand{\AttributeTok}[1]{\textcolor[rgb]{0.77,0.63,0.00}{#1}}
\newcommand{\RegionMarkerTok}[1]{#1}
\newcommand{\InformationTok}[1]{\textcolor[rgb]{0.56,0.35,0.01}{\textbf{\textit{#1}}}}
\newcommand{\WarningTok}[1]{\textcolor[rgb]{0.56,0.35,0.01}{\textbf{\textit{#1}}}}
\newcommand{\AlertTok}[1]{\textcolor[rgb]{0.94,0.16,0.16}{#1}}
\newcommand{\ErrorTok}[1]{\textcolor[rgb]{0.64,0.00,0.00}{\textbf{#1}}}
\newcommand{\NormalTok}[1]{#1}
\usepackage{graphicx,grffile}
\makeatletter
\def\maxwidth{\ifdim\Gin@nat@width>\linewidth\linewidth\else\Gin@nat@width\fi}
\def\maxheight{\ifdim\Gin@nat@height>\textheight\textheight\else\Gin@nat@height\fi}
\makeatother
% Scale images if necessary, so that they will not overflow the page
% margins by default, and it is still possible to overwrite the defaults
% using explicit options in \includegraphics[width, height, ...]{}
\setkeys{Gin}{width=\maxwidth,height=\maxheight,keepaspectratio}
\IfFileExists{parskip.sty}{%
\usepackage{parskip}
}{% else
\setlength{\parindent}{0pt}
\setlength{\parskip}{6pt plus 2pt minus 1pt}
}
\setlength{\emergencystretch}{3em}  % prevent overfull lines
\providecommand{\tightlist}{%
  \setlength{\itemsep}{0pt}\setlength{\parskip}{0pt}}
\setcounter{secnumdepth}{0}
% Redefines (sub)paragraphs to behave more like sections
\ifx\paragraph\undefined\else
\let\oldparagraph\paragraph
\renewcommand{\paragraph}[1]{\oldparagraph{#1}\mbox{}}
\fi
\ifx\subparagraph\undefined\else
\let\oldsubparagraph\subparagraph
\renewcommand{\subparagraph}[1]{\oldsubparagraph{#1}\mbox{}}
\fi

%%% Use protect on footnotes to avoid problems with footnotes in titles
\let\rmarkdownfootnote\footnote%
\def\footnote{\protect\rmarkdownfootnote}

%%% Change title format to be more compact
\usepackage{titling}

% Create subtitle command for use in maketitle
\newcommand{\subtitle}[1]{
  \posttitle{
    \begin{center}\large#1\end{center}
    }
}

\setlength{\droptitle}{-2em}

  \title{A Diaz, Guzman, \& Co. Xmas: Secret Santa 2018}
    \pretitle{\vspace{\droptitle}\centering\huge}
  \posttitle{\par}
    \author{David A. Diaz}
    \preauthor{\centering\large\emph}
  \postauthor{\par}
    \date{}
    \predate{}\postdate{}
  
\usepackage{booktabs}
\usepackage{longtable}
\usepackage{array}
\usepackage{multirow}
\usepackage[table]{xcolor}
\usepackage{wrapfig}
\usepackage{float}
\usepackage{colortbl}
\usepackage{pdflscape}
\usepackage{tabu}
\usepackage{threeparttable}
\usepackage{threeparttablex}
\usepackage[normalem]{ulem}
\usepackage{makecell}

\begin{document}
\maketitle

\section{Introduction}\label{introduction}

Hello and welcome familia Diaz/Guzman (+ all others)! We are excited to
continue our Secret Santa tradition for 2019. Let's go over some quick
guidelines and then delve in!

\begin{itemize}
\tightlist
\item
  As always, everyone is randomly matched to another human (or to a
  Cindy, a dog, or a cat).
\item
  Your mission is to buy a thoughtful gift for your Secret Santa to be
  given at our Christmas celebration on either December 24th or 25th.
\item
  The dollar limit for this year's big event is \textbf{\$15}.
\end{itemize}

\begin{center}\includegraphics{Secret_Santa_files/figure-latex/unnamed-chunk-1-1} \end{center}

\section{The Participants}\label{the-participants}

Let's see who is participating this year:

\begin{verbatim}
## readxl works best with a newer version of the tibble package.
## You currently have tibble v1.4.2.
## Falling back to column name repair from tibble <= v1.4.2.
## Message displays once per session.
\end{verbatim}

\begin{table}[H]
\centering
\begin{tabular}{l}
\hline
Name\\
\hline
Aaron Guzman\\
\hline
Abby Diaz\\
\hline
Adrian Guzman\\
\hline
Adriana Diaz\\
\hline
Ale Diaz\\
\hline
Ayden Guzman\\
\hline
Carlos Guzman\\
\hline
Cindy Diaz\\
\hline
Cookie Diaz\\
\hline
Dany Diaz\\
\hline
Estrella Diaz\\
\hline
Jony Guzman\\
\hline
Kelsey Hayes\\
\hline
Mama Letty\\
\hline
Mambo Guzman\\
\hline
Nando Guzman\\
\hline
Nella Guzman\\
\hline
Norma Guzman\\
\hline
Papa Lico\\
\hline
Rhina Guzman\\
\hline
Rigo Guzman\\
\hline
Rudy Sizer\\
\hline
Sarah Sizer\\
\hline
Sofia Escobar\\
\hline
Stella Diaz\\
\hline
Tio Juan\\
\hline
Tio Rigo\\
\hline
Tito Diaz\\
\hline
\end{tabular}
\end{table}

\section{The Matching}\label{the-matching}

An easy way to do the matching in \textbf{R} is to pick all the
participants in random order and assign them to give a gift to the
following person in the list. This helps to not have any self-matches
that might occur in a ``draw-a-name-out-of-a-hat'' approach, though it
does prevent any cross-gifting, i.e., no two people \emph{i} and
\emph{j} can be each other's Santa. I think that's actually a nice
feature - let's mix it up!

I'll show an example of how it works, but will use IDs instead of names
to keep it generic.

\begin{Shaded}
\begin{Highlighting}[]
\NormalTok{famID <-}\StringTok{ }\NormalTok{fam }\OperatorTok
\StringTok{       }\CommentTok{# Assign ID to each person}
\StringTok{       }\KeywordTok{mutate}\NormalTok{(}\DataTypeTok{ID =} \KeywordTok{sample}\NormalTok{(}\KeywordTok{seq}\NormalTok{(}\DecValTok{100}\NormalTok{,}\DecValTok{200}\NormalTok{), }\KeywordTok{length}\NormalTok{(fam}\OperatorTok{$}\NormalTok{Name), }\DataTypeTok{replace=}\OtherTok{FALSE}\NormalTok{)) }\OperatorTok\StringTok{ }
\StringTok{       }\KeywordTok{select}\NormalTok{(ID) }\OperatorTok
\StringTok{       }\CommentTok{# Randomly draw every ID }
\StringTok{       }\KeywordTok{mutate}\NormalTok{(}\DataTypeTok{ID =} \KeywordTok{sample}\NormalTok{(ID)) }\OperatorTok\StringTok{     }
\StringTok{      }\CommentTok{# Pick the next ID in list as their partner}
\StringTok{       }\KeywordTok{mutate}\NormalTok{(}\DataTypeTok{Partner_ID =} \KeywordTok{lag}\NormalTok{(ID))  }

\CommentTok{# Assign remaining partner to first person in list}
\NormalTok{famID}\OperatorTok{$}\NormalTok{Partner_ID[}\DecValTok{1}\NormalTok{] <-}\StringTok{ }\NormalTok{famID}\OperatorTok{$}\NormalTok{ID[}\KeywordTok{nrow}\NormalTok{(famID)]                               }

\CommentTok{# Show table}
\NormalTok{famID }\OperatorTok\StringTok{ }\KeywordTok{kable}\NormalTok{(}\DataTypeTok{align=}\StringTok{'l'}\NormalTok{) }\OperatorTok\StringTok{ }\KeywordTok{kable_styling}\NormalTok{(}\DataTypeTok{bootstrap_options =} \StringTok{"striped"}\NormalTok{)}
\end{Highlighting}
\end{Shaded}

\begin{table}[H]
\centering
\begin{tabular}{l|l}
\hline
ID & Partner\_ID\\
\hline
192 & 138\\
\hline
161 & 192\\
\hline
146 & 161\\
\hline
196 & 146\\
\hline
109 & 196\\
\hline
111 & 109\\
\hline
122 & 111\\
\hline
183 & 122\\
\hline
165 & 183\\
\hline
108 & 165\\
\hline
175 & 108\\
\hline
181 & 175\\
\hline
185 & 181\\
\hline
133 & 185\\
\hline
168 & 133\\
\hline
184 & 168\\
\hline
105 & 184\\
\hline
141 & 105\\
\hline
200 & 141\\
\hline
147 & 200\\
\hline
139 & 147\\
\hline
153 & 139\\
\hline
121 & 153\\
\hline
120 & 121\\
\hline
164 & 120\\
\hline
151 & 164\\
\hline
125 & 151\\
\hline
138 & 125\\
\hline
\end{tabular}
\end{table}

\section{The Final Results}\label{the-final-results}

The code that generates the final Secret Santa matches is shown below.
Note that this will generate a new partner each time it is run - so
don't try it at home and expect the partners to be the same as in the
final run!

\begin{Shaded}
\begin{Highlighting}[]
\CommentTok{# Set seed for reproducing matches - good luck guessing my number :)}
\CommentTok{# set.seed()}

\NormalTok{fam <-}\StringTok{ }\NormalTok{fam }\OperatorTok
\StringTok{       }\KeywordTok{select}\NormalTok{(Name) }\OperatorTok
\StringTok{       }\KeywordTok{mutate}\NormalTok{(}\DataTypeTok{Name =} \KeywordTok{sample}\NormalTok{(Name)) }\OperatorTok\StringTok{     }
\StringTok{       }\KeywordTok{mutate}\NormalTok{(}\DataTypeTok{Partner =} \KeywordTok{lag}\NormalTok{(Name))  }
       
\NormalTok{fam}\OperatorTok{$}\NormalTok{Partner[}\DecValTok{1}\NormalTok{] <-}\StringTok{ }\NormalTok{fam}\OperatorTok{$}\NormalTok{Name[}\KeywordTok{nrow}\NormalTok{(fam)]  }
\end{Highlighting}
\end{Shaded}


\end{document}
